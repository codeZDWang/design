%!TEX root = ../Demo.tex
% 中文摘要
\begin{abstract}
随着云计算技术的发展,虚拟化技术已经成为提高云平台资源利用率的有效手段。随着数据中心的服务器不断运行,服务器中虚拟机会不断的开启、关闭,导致某些服务器负载过重,某些服务器却空闲。通过虚拟机的动态迁移技术,将高负载服务器上运行的虚拟机迁移至空闲或低负载服务器上,从而实现服务器的负载均衡。

本文根据蚁群算法的思想设计了一种面向负载均衡的虚拟机动态迁移框架。首先进行框架的总体设计,然后对框架的各个模块进行详细地设计,其中蚂蚁算法模块是框架的核心,其核心思想是利用蚂蚁在搜索过程中获取并记录服务器信息,根据信息素等信息不断优化搜索路径,找到适合迁移的服务器。通过设置不同的阙值来定义一个服务器的负载状态模型,通过定时计算服务器的负载值来动态判断服务器所处的负载域,当服务器过载该服务器便会成为待迁移的服务器,从而生成蚂蚁去搜寻可以迁移的目的服务器。最终通过设计选择策略和迁移策略来选择待迁移服务器上适合迁移的虚拟机和虚拟机适合迁移到的目的服务器,最终完成虚拟机的迁移过程。一个数据中心内的所有服务器,不断重复这一迁移过程来达到整个数据中心内的服务器的负载均衡。

最后通过扩展云计算建模和仿真框架CloudSim,应用本文提出的虚拟机迁移框架进行虚拟机迁移的模拟实验,最终对实验结果进行分析与评估,验证了本文设计的虚拟机迁移框架能够有效地提高数据中心的负载均衡程度。

\end{abstract}
\keywords{云计算, 虚拟化, 虚拟机迁移, 负载均衡}

% 英文摘要
\begin{enabstract}
With the development of cloud computing technology, virtualization technology has become an effective means to improve the utilization of cloud platform resources. As the servers in the data center continue to run, the virtual machines in the server are constantly turned on and off, causing some servers to be overloaded and some servers to be idle. The virtual machine's dynamic migration technology is used to migrate the virtual machines running on the high-load servers to idle or low-load servers to achieve load balancing of the servers.

According to the idea of ​​ant colony algorithm, this paper designs a virtual machine dynamic migration framework for load balancing. Firstly, the overall design of the framework is carried out, and then the various modules of the framework are designed in detail. The ant algorithm module is the core of the framework. The core idea is to use ants to acquire and record server information during the search process, and optimize according to information such as pheromone. Search for the path and find the server that is suitable for migration. Define a load state model of a server by setting different thresholds. Dynamically determine the load domain where the server is located by periodically calculating the load value of the server. When the server is overloaded, the server becomes the server to be migrated, thereby generating ants to search. The destination server that can be migrated. Finally, the design selection strategy and the migration strategy are used to select the virtual machine and the virtual machine suitable for migration on the server to be migrated, and the migration process to the virtual machine is completed. All servers in a data center continue to replicate this migration process to achieve load balancing across servers throughout the data center.

Finally, by extending the cloud computing modeling and simulation framework CloudSim, the virtual machine migration framework proposed in this paper is used to simulate the virtual machine migration. Finally, the experimental results are analyzed and evaluated, and the virtual machine migration framework designed in this paper can be effectively Increase the load balancing of your data center.

\end{enabstract}
\enkeywords{cloud computing, virtualization virtual, machine migration, load balancing}